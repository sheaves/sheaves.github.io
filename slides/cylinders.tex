\documentclass[compress]{beamer}
%\documentclass[notes]{beamer}
\usepackage[utf8]{inputenc}
 
% Theming

	\usetheme{Frankfurt}
	% Remove header bar
	\setbeamertemplate{headline}{}		

	% Turn off navigation symbols
	\setbeamertemplate{navigation symbols}{}

	% Table of contents at every section
	\AtBeginSection[]
	{
	  \begin{frame}<beamer>{Outline}
	    \tableofcontents[currentsection]
	  \end{frame}
	}

	% Citation
	% https://tex.stackexchange.com/a/6015/149868
	% \usepackage[backend=bibtex]{biblatex}
	% \bibliography{biblio}		
	% \setbeamerfont{footnote}{size=\tiny}

	% Text color
	\usepackage{xcolor}

	% For strike through
	% https://tex.stackexchange.com/a/23712/149868
	\usepackage[normalem]{ulem}

	% Table settings
	%\setlength{\arrayrulewidth}{1mm}
	\setlength{\tabcolsep}{18pt}
	\renewcommand{\arraystretch}{1.5}	

	\usepackage{array}

	\usepackage{csquotes}

	% Checks and Crosses
	\usepackage{pifont}% http://ctan.org/pkg/pifont
	\newcommand{\cmark}{\ding{51}}%
	\newcommand{\xmark}{\ding{55}}%

% Math macros
	\usepackage{amsmath,amssymb,amsthm}
	\usepackage{mathtools}
	\let\amssquare\square

	\usepackage{bbm} %for blackboard bold numbers

	% for Warning sign: 
	% https://tex.stackexchange.com/questions/159669/how-to-print-a-warning-sign-triangle-with-exclamation-point
	% \usepackage{fontawesome}
	% \usepackage{fourier}

	\usepackage{tikz-cd}
	\tikzset{ % "visible on" tag
	    invisible/.style={opacity=0},
	    visible on/.style={alt={#1{}{invisible}}},
	    alt/.code args={<#1>#2#3}{%
	      \alt<#1>{\pgfkeysalso{#2}}{\pgfkeysalso{#3}}%
	  }
	}
	\tikzset{% for drawing adjunctions
	    symbol/.style={%
	        draw=none,
	        every to/.append style={%
	            edge node={node [sloped, allow upside down, auto=false]{$#1$}}}
	    }
	}

	\usepackage{tikz}

	\usetikzlibrary{decorations.pathmorphing} % for squiggly

	%% Theorem environments
		% http://www.maths.tcd.ie/~dwilkins/LaTeXPrimer/Theorems.html
		% \newtheorem{theorem}{Theorem}[section]
		% \newtheorem{lemma}[theorem]{Lemma}
		\newtheorem{proposition}[theorem]{Proposition}
		% \newtheorem{corollary}[theorem]{Corollary}

		% % \newenvironment{proof}[1][Proof]{\begin{trivlist}
		% % \item[\hskip \labelsep {\bfseries #1}]}{\end{trivlist}}
		% \newenvironment{definition}[1][Definition]{\begin{trivlist}
		% \item[\hskip \labelsep {\bfseries #1}]}{\end{trivlist}}
		% \newenvironment{example}[1][Example]{\begin{trivlist}
		% \item[\hskip \labelsep {\bfseries #1}]}{\end{trivlist}}
		% \newenvironment{remark}[1][Remark]{\begin{trivlist}
		% \item[\hskip \labelsep {\bfseries #1}]}{\end{trivlist}}


	%%% Categories
		\newcommand{\cat}[1]{\mathbf{#1}}
		\newcommand{\Set}{\cat{Set}}
		\newcommand{\sSet}{\cat{sSet}}
		\newcommand{\cSet}{\cat{cSet}}
		\newcommand{\Fun}{\cat{Fun}}

	%%% Variable categories
		\newcommand{\varcat}[1]{\mathcal{#1}}
		\newcommand{\A}{\varcat{A}}
		\newcommand{\B}{\varcat{B}}
		\newcommand{\C}{\varcat{C}}
		\newcommand{\D}{\varcat{D}}
		\newcommand{\E}{\varcat{E}}

		\newcommand{\Q}{\mathcal{Q}}
		
		\renewcommand{\L}{\mathcal{L}}
		\newcommand{\R}{\mathcal{R}}		

		\newcommand{\I}{\mathbb{I}}		

		\newcommand{\To}{\Rightarrow}

		\newcommand{\N}{\mathsf{N}}

		% \newcommand{\ii}{{\scriptscriptstyle \cdot \to \cdot}}
		% \newcommand{\iii}{{\scriptscriptstyle \cdot\to \cdot \to \cdot}}
		% \newcommand{\ii}{\mathbf{2}}
		% \newcommand{\iii}{\mathbf{3}}
		\newcommand{\ii}{{\scriptscriptstyle [1]}}
		\newcommand{\iii}{{\scriptscriptstyle [2]}}

	%%% Tildes
		\renewcommand{\t}[1]{\tilde{#1}}

	%%% Others

		\newcommand{\Cube}{\amssquare} % the cube category

		\newcommand{\op}{\mathsf{op}}
		\newcommand{\1}{\mathbf{1}}
		\newcommand{\cod}{\cat{cod}}
		\newcommand{\Id}{\mathsf{Id}}
		\newcommand{\Ob}{\mathsf{Ob}}
		\newcommand{\Hom}{\mathsf{Hom}}	
		\newcommand{\Ho}{\mathsf{Ho}}

		\newcommand{\Cyl}{\mathsf{Cy}}

		\newcommand{\xto}[1]{\xrightarrow{#1}}

		\newcommand{\coCart}{\cat{coCart}}




%Information to be included in the title page:
\title{Cubes with connections \\ from algebraic weak factorization systems}
\author{Liang Ze Wong \\ (work in progress with Chris Kapulkin)}
%\institute{}
\date{Western Topology Seminar\\ 11 Nov 2020}

\begin{document}
 
\frame{\titlepage}

\begin{frame}{Outline}
	\begin{itemize}  \setlength{\itemsep}{15pt}
		\item The interval, homotopies, and cubes
		\item Weak factorization systems
			\begin{itemize}
				\item Functorial
				\item Algebraic
			\end{itemize}
		\item Cylinders from functorial and algebraic WFS
	\end{itemize}
\end{frame}

\begin{frame}[fragile]{The Interval and Homotopies}
	The interval $\I = [0,1]$ allows us to define \emph{homotopies}: 
	\pause \vfill
	\begin{block}{}
	A homotopy from $f$ to $g$ is a function $H$ fitting into the diagram:
	\[
		\begin{tikzcd}[sep = large]
			X \ar[dr, "i_0"'] \ar[drr, bend left, "f"] & &
			\\
			& X \times \I \ar[r, "H"] & Y
			\\
			X \ar[ur, "i_1"] \ar[urr, bend right, "g"'] & &
		\end{tikzcd}
	\]
	\end{block}
	\pause
	From there, we can define \emph{homotopy equivalences} between objects:
	\pause
	\vfill
	$\to$ first example of \emph{weak equivalences} encountered in math
\end{frame}

\begin{frame}[fragile]{The Interval and Higher-dimensional Geometry}
	The interval is also the $1$-dim version of various $n$-dim objects:
	\pause \vfill
	\begin{centering}
		  \includegraphics[width=\linewidth]{tetracubeball.png}
	\end{centering}
	\pause \vfill
	Today, we will focus on \emph{cubes}.
\end{frame}

\begin{frame}{Why cubes?}
	\begin{proposition}
		Higher-dimensional cubes are more appealing and intuitive than simplices and balls.
	\end{proposition}
	\pause
	\begin{block}{Proof.}
	\centering
		  \includegraphics[width=0.75\linewidth]{tesseract.png}
	\end{block}
\end{frame}

\begin{frame}[fragile]{Simplices vs cubes}
	The \emph{simplex category} $\Delta$ does have the advantage of simplicity:
	\begin{itemize}
		\item objects are $[n] = \{0 \leq 1 \leq \dots \leq n \}$
		\item morphisms are all order-preserving maps
			\begin{itemize}
				\item generated by faces and degeneracies
			\end{itemize}
	\end{itemize}	
	\pause
	\vfill
	By contrast, the \emph{cube category} $\Cube$ requires some choices:
	\begin{itemize}
		\item objects are $[1]^n = \{0 \leq 1\}^n$
		\item morphisms are {\color{red}some subset of} order-preserving maps
			\begin{itemize}
				\item generated by faces and degeneracies, and possibly $\dots$
				\item connections
				\item symmetries
				\item diagonals
			\end{itemize}
	\end{itemize}	
\end{frame}

\begin{frame}[fragile]{Simplices vs cubes}
	Generators for maps in $\Cube$:
	\pause
	\begin{itemize}[<+->]
		\item \emph{face} and \emph{degeneracy} maps 
		\item \emph{connections} (max \& min)
		\item \emph{diagonals} and \emph{symmetries}
	\end{itemize}
	\vfill
	\[
		\begin{tikzpicture}		
			\node (00) at (-0.5,0.5) {$\cdot$};
			\node  (01) at (1.5,0.5) {$\cdot$};
			\node  (10) at (-0.5,2.5) {$\cdot$};
			\node (11) at (1.5,2.5) {$\cdot$};
			\draw[->] (00) -- (01);
			\draw[->] (00) -- (10);
			\draw[->] (01) -- (11);
			\draw[->] (10) -- (11);

			% Faces and degeneracies
			\node (0) at (-2.5,0.5) {{\visible<2>{$\cdot$}}};
			\node (1) at (-2.5,2.5) {{\visible<2>{$\cdot$}}};
			\draw[->, visible on = <2>] (0) -- (1);

			\node (0') at (-0.5,-1.5) {{\visible<2>{$\cdot$}}};
			\node (1') at (1.5,-1.5) {{\visible<2>{$\cdot$}}};
			\draw[->, visible on = <2>] (0') -- (1');	

			\draw[->, visible on = <2>, blue] (-2,1.75) -- ++(1,0);
			\draw[<-, visible on = <2>, green] (-2,1.25) -- ++(1,0);
			\draw[->, visible on = <2>, blue] (0.25,-1) -- ++(0,1);
			\draw[<-, visible on = <2>, green] (0.75,-1) -- ++(0,1);		

			% Connections
			\node (c0) at (1.5,-1.5) {{\visible<3-4>{$\cdot$}}};
			\node (c1) at (3.5,0.5) {{\visible<3-4>{$\cdot$}}};
			\draw[->, visible on = <3-4>] (c0) -- (c1);

			\draw[red, dashed, visible on = <3>]  plot[smooth, tension=.4] coordinates {(-0.5,3) (-1,2.5) (1.5,0) (2,0.5) (2,2.5) (1.5,3) (-0.5,3)};
			\draw[->, red, visible on = <3>] (00) -- (c0);
			\draw[<-, red, visible on = <3>] (c1) -- ++(-1.5,1.5);	

			% Symmetries
			\draw[<->, brown, visible on = <5>] (10.south east) .. controls (1,2) .. (01.north west);

			% Diagonals
			\draw[->, dashed, visible on = <4>] (00) -- (11);
			\draw[->, brown, visible on = <4>] (2.25,-0.25) .. controls (2,1) .. (0.75,1.25);						
		\end{tikzpicture}
	\]
\end{frame}

\begin{frame}{Cubes with connections}
	The case for faces, degeneracies \emph{and connections}: \\~\\
	\pause
	\begin{itemize}[<+->] \setlength{\itemsep}{15pt}
		\item $\Cube$ is a generalized Reedy category
		\item $\Cube$ is a strict test category
		\item There is a co-reflective embedding\footnote{Kapulkin-Lindsey-W. 2019} $\sSet \hookrightarrow \cSet$ 
		\item It's exactly what we get from \emph{algebraic} weak factorization systems (this talk)
	\end{itemize}
\end{frame}

\begin{frame}[fragile]{Interlude: Embedding $\sSet$ into $\cSet$}
	As long as $\Cube$ has faces and degeneracies, we can define certain quotients of the standard cubes in $\cSet$:
	\pause
	\begin{tikzpicture}
		\node at (-4,6) {$Q^0$};
		\node at (-3,6) {$=$};
		\node at (-2,6) {$\cdot$};

		\node at (1.75,6) {$Q^1$};
		\node at (2.75,6) {$=$};
		\node (v13) at (4,6) {$\cdot$};
		\node (v14) at (6,6) {$\cdot$};
		\draw[->]  (v13) edge (v14);

		\node at (-4,4) {$Q^2$};
		\node at (-3,4) {$=$};

		\node (v9) at (-1,3) {$\cdot$};
		\node (v10) at (-1,5) {$\cdot$};
		\node (v11) at (1,3) {$\cdot$};
		\node (v12) at (1,5) {$\cdot$};
		\draw[->]  (v9) edge (v10);
		\draw[->]  (v9) edge (v11);
		\draw  (v11) edge[double] (v12);
		\draw[->]  (v10) edge (v12);

		\node at (2.75,4) {$=$};

		\node (w9) at (4,3) {$\cdot$};
		\node (w10) at (4,5) {$\cdot$};
		\node (w11) at (6,3.75) {$\cdot$};
		\node (w12) at (6,4.25) {$\cdot$};
		\draw[->]  (w9) edge (w10);
		\draw[->]  (w9) edge (w11);
		\draw  (w11) edge[double] (w12);
		\draw[->]  (w10) edge (w12);

		\node at (-4,0.5) {$Q^3$};
		\node at (-3,0.5) {$=$};

		\node (v1) at (-1.5,-1) {$\cdot$};
		\node (v2) at (-1.5,1) {$\cdot$};
		\node (v5) at (0.5,1) {$\cdot$};
		\node (v3) at (0.5,-1) {$\cdot$};
		\node (v4) at (-0.5,2) {$\cdot$};
		\node (v6) at (1.5,2) {$\cdot$};
		\node (v7) at (1.5,0) {$\cdot$};
		\node (v8) at (-0.5,0) {$\cdot$};
		\draw[->]  (v1) edge (v2);
		\draw[->]  (v1) edge (v3);
		\draw  (v2) edge[double] (v4);
		\draw[->, red]  (v2) edge (v5);
		\draw[->, red]  (v4) edge (v6);
		\draw  (v5) edge[double] (v6);
		\draw  (v3) edge[double] (v5);
		\draw  (v3) edge[double] (v7);
		\draw  (v7) edge[double] (v6);
		\draw[->]  (v1) edge (v8);
		\draw[->]  (v8) edge (v7);
		\draw[->]  (v8) edge (v4);
		\path[fill=blue, opacity = 0.2] (v5.center) -- (v6.center) -- (v7.center) -- (v3.center) -- (v5.center);

		\node at (2.75,0.5) {$=$};

		\node (w1) at (4,-1) {$\cdot$};
		\node (w2) at (4.5,1.5) {$\cdot$};
		\node (w5) at (6.5,0) {$\cdot$};
		\node (w3) at (6.5,-0.5) {$\cdot$};
		\node (w4) at (4.75,1.75) {$\cdot$};
		\node (w6) at (6.75,0.25) {$\cdot$};
		\node (w7) at (6.75,-0.25) {$\cdot$};
		\node (w8) at (5,0) {$\cdot$};
		\draw[->]  (w1) edge (w2);
		\draw[->]  (w1) edge (w3);
		\draw  (w2) edge[double] (w4);
		\draw[->, red]  (w2) edge (w5);
		\draw[->, red]  (w4) edge (w6);
		\draw  (w5) edge[double] (w6);
		\draw  (w3) edge[double] (w5);
		\draw  (w3) edge[double] (w7);
		\draw  (w7) edge[double] (w6);
		\draw[->]  (w1) edge (w8);
		\draw[->]  (w8) edge (w7);
		\draw[->]  (w8) edge (w4);
		\path[fill=blue, opacity = 0.2] (w5.center) -- (w6.center) -- (w7.center) -- (w3.center) -- (w5.center);
	\end{tikzpicture}
\end{frame}

\begin{frame}[fragile]{Interlude: Embedding $\sSet$ into $\cSet$}
	But we need connections to get all the "simplicial degeneracies" between $Q^n$s:
	\pause
	\[
		\begin{tikzpicture}
			% Delta 2
			\node(1) at (-0.5,2) {$\cdot$};
			\node(0) at (-2,-0.5) {$\cdot$};
			\node(2) at (1,-0.5) {$\cdot$};
			\node(2x) at (1,0) {$\cdot$};

			\draw[->] (0) -- (1);
			\draw[->] (0) -- (2);
			\draw[->] (1) -- (2x);		
			\draw (2) edge[double] (2x);			

			% Delta 1
			\node(0') at (-2,-3) {$\cdot$};
			\node(1') at (1,-3) {$\cdot$};
			\draw[->] (0') -- (1');
		
			% Degeneracy maps
			\node(c1) at (-1.25,0.75) {};
			\draw[rotate around = { -30: (c1) }, dashed, green, visible on = <3>] (c1) ellipse (0.5 and 2);
			\draw[<-, green, visible on = <3>] (0') -- ++(0,2);
			\draw[->, green, visible on = <3>] (2) -- (1');

			\node(c2) at (0.25,0.75) {};
			\draw[rotate around = {30 : (c2) }, dashed, red, visible on = <5->] (c2) ellipse (0.5 and 2);
			\draw[<-, red, visible on = <5->] (1') -- ++(0,2);	
			\draw[->, red, visible on = <5->] (0) -- (0');
		\end{tikzpicture}		
	\]	
\end{frame}

\begin{frame}[fragile]{So far...}
	\[
		\begin{tikzcd}[column sep = small]
			& \text{the interval} \ar[dl] \ar[dr] &
			\\
			\text{classical homotopy theory} \ar[d, rightsquigarrow, visible on = <3->] & & \text{cubes} \ar[d, rightsquigarrow, visible on = <2->]
			\\
			|[visible on=<3->]|\text{abstract homotopy theory} \ar[rr,  dashed, no head, "?" description, visible on = <4->] & & |[visible on=<2->]|\text{the cube category}
			\\[-20pt]
			|[visible on=<3->]|\text{(model categories)}
		\end{tikzcd}
	\]
\end{frame}

% \begin{frame}{Abstract homotopy theory}
% 	Some approaches to abstract homotopy theory: \\~\\
% 	\begin{itemize} \setlength{\itemsep}{15pt}
% 		\item model categories
% 		\item (co)fibration categories
% 		\item categories with (co)cylinders
% 	\end{itemize}
% 	\pause \vfill
% 	Many of these involve some notion of \emph{factorization} and/or of \emph{(co)cylinders}.
% \end{frame}

\begin{frame}[fragile]{Model categories}
	A \emph{model category} has all limits and colimits, and classes of: \\~\\
	\begin{itemize} \setlength{\itemsep}{15pt}
		\item $\begin{tikzcd} \phantom{\cdot} \ar[r, "\sim"] & \phantom{\cdot} \end{tikzcd}$ \emph{weak equivalences}  satisfying 2-out-of-3
		\item  $\begin{tikzcd} \phantom{\cdot} \ar[r, rightarrowtail] & \phantom{\cdot} \end{tikzcd}$ \emph{cofibrations} (nice injections)
		\item  $\begin{tikzcd} \phantom{\cdot} \ar[r, two heads] & \phantom{\cdot} \end{tikzcd}$ \emph{fibrations} (nice surjections)
	\end{itemize}
	\pause \vfill
	such that we have \emph{weak factorization systems}:
	\[
		\begin{tikzcd}[row sep = tiny]
			(\phantom{\cdot} \ar[r, rightarrowtail, "\sim"] & \phantom{\cdot},
			 \phantom{\cdot} \ar[r, two heads] & \phantom{\cdot})
			 \\
			(\phantom{\cdot} \ar[r, rightarrowtail] & \phantom{\cdot},
			 \phantom{\cdot} \ar[r, two heads, "\sim"] & \phantom{\cdot})
		\end{tikzcd}
	\]
\end{frame}

\begin{frame}[fragile]{Weak Factorization Systems}
	A \emph{weak factorization system (WFS)} on a category $\C$ consists of two classes of maps $\L, \R$ such that:
	\pause
	\begin{itemize}[<+->]
		\item Every map in $\C$ factors as 
		\[
			\begin{tikzcd}
				\cdot \ar[r, "\in\; \L"] & \cdot \ar[r, "\in \; \R"] & \cdot
			\end{tikzcd}
		\]
		\item There exist lifts for squares of the following form:
		\[
		\begin{tikzcd}[sep = large]
			\cdot \ar[r] \ar[d, "\L \; \ni"'] & \cdot \ar[d, "\in \; \R"]
			\\
			\cdot \ar[r] \ar[ur, dashed, "\exists", red] & \cdot
		\end{tikzcd}
		\]
	\end{itemize}
	\pause
	Factorizations and lifts are \emph{not unique}!
\end{frame}

\begin{frame}[fragile]{Weak Factorization Systems}
	An \emph{orthogonal} factorization system is one where lifts are unique:
		\[
		\begin{tikzcd}[sep = large]
			\cdot \ar[r] \ar[d, "\L \; \ni"'] & \cdot \ar[d, "\in \; \R"]
			\\
			\cdot \ar[r] \ar[ur, dashed, "\exists\, !", red] & \cdot
		\end{tikzcd}
		\]
	\pause
	But OFS are hard to come by in homotopical contexts.
	\pause \vfill

	Intermediate versions:
	\[
		\begin{tikzcd}[row sep = tiny]
			\cdot \ar[r, no head] & \cdot \ar[r, no head] & \cdot\ar[rr, no head] & & \cdot
			\\
			\text{weak} & {\color{red}\text{functorial}} & {\color{red}\text{algebraic}} & & \text{orthogonal}
		\end{tikzcd}
	\]
\end{frame}

\begin{frame}[fragile]{Weak Factorization Systems}
	\begin{tabular}{r | c c}
		& Factorizations & Lifts 
		\\ \hline
		weak & exist & exist 
		\\ 
		functorial & \onslide<2->{functorial}  & \onslide<2->{canonical} 
		\\
		algebraic & \onslide<3->{functorial}  & \onslide<3->{*} 
		\\
		orthogonal & unique & unique
		\\[-8pt]
		& {\scriptsize (up to iso)} &
	\end{tabular}	
	\vfill
	\onslide<3->{* Uniqueness of lifts of a very specific kind}
	\vfill
	\onslide<4->{We will see how to get cylinders from these WFS}
\end{frame}

\begin{frame}[fragile]{Cylinders from WFS}
	From now on, assume that $\C$ has coproducts and a WFS:
	\[
		\begin{tikzcd}[row sep = tiny]
			(\phantom{\cdot} \ar[r, rightarrowtail] & \phantom{\cdot},
			 \phantom{\cdot} \ar[r, two heads, "\sim"] & \phantom{\cdot})
		\end{tikzcd}
	\]
	\pause \vfill
	For each $X \in \C$, there is the \emph{co-diagonal} or \emph{fold} map:
	\[
		X \sqcup X \xto{\quad \nabla \quad} X
	\]
	\pause \vfill
	Factoring $\nabla$ gives a \emph{cylinder object} for $X$:
	\[
		\begin{tikzcd}[column sep = large]
			X \sqcup X \ar[r, rightarrowtail] & \Cyl X \ar[r, two heads, "\sim"] & X
		\end{tikzcd}
	\]
	\pause \vfill
	$\Cyl X$ is a substitute for $X \otimes \I$ in the absence of $\I$ or $\otimes$
	\pause 

	e.g. can define homotopies using $\Cyl X$
\end{frame}

\begin{frame}[fragile]{Cylinders from functorial WFS}
	If we have a \emph{functorial} WFS, then $\Cyl$ becomes a functor as well:
	\[	
		\begin{tikzcd}
			X \sqcup X \ar[r, "f \sqcup f"] \ar[d, "\nabla"'] & Y \sqcup Y \ar[d, "\nabla"]
			\\
			X \ar[r, "f", red] \ar[r, "f", visible on = <2->] & Y			
		\end{tikzcd}
		\quad \quad \mapsto \quad \quad
		\begin{tikzcd}[row sep = large]
			X \sqcup X \ar[r, "f \sqcup f"] \ar[d, rightarrowtail] \ar[d, rightarrowtail, "\partial_0 \sqcup \partial_1"', red, visible on = <2->] & Y \sqcup Y \ar[d, rightarrowtail] \ar[d, rightarrowtail, "\partial_0 \sqcup \partial_1", red, visible on = <2->]
			\\
			\Cyl X \ar[r, "\Cyl f", red]\ar[r, "\Cyl f", visible on = <2->] \ar[d, two heads, "\sim"] \ar[d, two heads, "\sim", "\sigma"', red, visible on = <2->] & \Cyl Y \ar[d, two heads, "\sim"'] \ar[d, two heads, "\sim"', "\sigma", red, visible on = <2->]
			\\
			X \ar[r, "f"] & Y
		\end{tikzcd}
	\]
	\pause
	And the left and right factorizations of $\nabla$ are components of \emph{face} and \emph{degeneracy} natural transformations:
	\[
		\partial_i \colon \Id \To \Cyl \quad \quad \quad \quad \sigma \colon \Cyl \To \Id
	\]
\end{frame}

\begin{frame}[fragile]{Cylinders from functorial WFS}
	Applying $\Cyl$ to the components of $\partial_i$ and $\sigma$, we get the \emph{cubical identities} for faces and degeneracies:
	\[
		\begin{tikzcd}[sep = large]
			X \sqcup X \ar[r, "\partial_i \sqcup \partial_i"] \ar[d, "\partial_0 \sqcup \partial_1"'] & \Cyl X \sqcup \Cyl X \ar[r, "\sigma \sqcup \sigma"] \ar[d, "\partial_0 \circ \Cyl \sqcup \partial_1 \circ \Cyl" description] & X \sqcup X \ar[d, "\partial_0 \sqcup \partial_1"]
			\\
			\Cyl X \ar[r, "\Cyl \circ \partial_i"] \ar[d, "\sigma"'] & \Cyl^2 X \ar[r, "\Cyl \circ \sigma"] \ar[d, "\sigma \circ \Cyl" description] & \Cyl X \ar[d, "\sigma"]
			\\
			X \ar[r, "\partial_i"] & \Cyl \ar[r, "\sigma"] X & X
		\end{tikzcd}
	\]	
	\begin{align*}
		\sigma\; \partial_i &= \Id \\
		(\partial_i \circ \Cyl)\; \partial_j &= (\Cyl \circ \partial_j)\; \partial_i \\
		(\sigma \circ \Cyl) (\Cyl \circ \partial_i) = &\; \partial_i \;\sigma = (\Cyl \circ \sigma)(\partial_i \circ \Cyl) \\
		\sigma\;(\Cyl \circ \sigma) &= \sigma\; (\sigma \circ \Cyl)
	\end{align*}	
\end{frame}

\begin{frame}[fragile]{Cylinders from functorial WFS}
	\begin{lemma}
		A category $\C$ with coproducts and a functorial WFS has:
		\begin{itemize}
			\item a cylinder functor $\Cyl \colon \C \to \C$
			\item faces $\partial_0, \partial_1 \colon \Id \To \Cyl$ 
			\item degeneracies $\sigma \colon \Cyl \To \Id$ 
		\end{itemize}
		satisfying the cubical identities for faces and degeneracies.
	\end{lemma}
	\pause \vfill
	Each $X \in \C$ gives a co-cubical object (without connections):
	\[
		\begin{tikzcd}
			X \ar[r, shift left = 2] \ar[r, shift right = 2] & \Cyl X \ar[l] \ar[r, shift left = 2] \ar[r, shift left = 6] \ar[r, shift right = 2] \ar[r, shift right = 6] & \Cyl^2 X \ar[l, shift left = 4] \ar[l, shift right = 4] \ar[r, phantom, "\cdots"] & \cdots
		\end{tikzcd}
	\]
\end{frame}

\begin{frame}[fragile]{Cylinders from functorial WFS}
	With the current setup (coproducts and functorial WFS), we can actually define connections:
	\[
		\begin{tikzpicture}[scale = 0.6]
			\fill[lightgray](-4.9,-2.1) rectangle (-3.1,-3.9);
			
			\node (00) at (-5,2.5) {$\bullet$};
			\node  (01) at (-3,2.5) {$\circ$};
			\node  (10) at (-5,4.5) {$\circ$};
			\node (11) at (-3,4.5) {$\circ$};
			
			\draw[-] (00) -- (10);
			\draw[-] (01) -- (11);


			\node (00) at (2,2.5) {$\bullet$};
			\node (11) at (4,4.5) {$\circ$};
			\draw[-] (00) -- (11);

			\node (00) at (-5,-4) {$\bullet$};
			\node  (01) at (-3,-4) {$\circ$};
			\node  (10) at (-5,-2) {$\circ$};
			\node (11) at (-3,-2) {$\circ$};
			\draw[-] (00) -- (01);
			\draw[-] (00) -- (10);
			\draw[-] (01) -- (11);
			\draw[-] (10) -- (11);	

			\node (00) at (3,-3) {$\bullet$};
			
			\draw[>->] (-4,1.5) -- (-4,-1);
			\draw[->>] (-2,-3) -- (1,-3) node[midway,above] {$\sim$};
			\draw[->>] (3,1.5) -- (3,-1) node[midway,right] {$\sim$};
			\draw[->] (-2,3.5) -- (1,3.5);
			
			\pause
			\draw[green, dashed]  plot[smooth, tension=.4] coordinates {(-5,5) (-5.5,4.5) (-3,2) (-2.5,2.5) (-2.5,4.5) (-3,5) (-5,5)};
			\draw[->, green] (-2.5,5) .. controls (0.5,5.5) .. (3.75,4.75);
			
			\pause
			\draw[->, red, dashed] (-2.5,-1.5) -- (1.5,2) node[midway,above] {$\exists\,\gamma$};	
		\end{tikzpicture}
	\]
	\onslide<4->{But we can't get some cubical identities without uniqueness of lifts.}
\end{frame}

\begin{frame}[fragile]{Cylinders from algebraic WFS}
	In a functorial WFS, we have $L \colon \C^\ii \to \C^\ii$ and $R \colon \C^\ii \to \C^\ii$:
	\[
		\begin{tikzcd}	
			X \ar[r, "f"] & Y
		\end{tikzcd}
		\quad \quad
		\rightsquigarrow
		\quad \quad
		\begin{tikzcd}
			X \ar[r, "Lf"] & \cdot \ar[r, "Rf"] & Y
		\end{tikzcd}
	\]
	\pause
	In an \emph{algebraic} WFS, $L$ is a \emph{comonad} and $R$ is a \emph{monad}. \\~\\
	\pause 
	Although we still don't have uniqueness of lifts in general, we have uniqueness of $R$-algebra homomorphisms out of \emph{free $R$-algebras}. \\~\\
	\pause
	$\to$ this is enough to make connections satisfy cubical identities!
\end{frame}

\begin{frame}[fragile]{Cylinders from algebraic WFS}
	\begin{proposition}[Kapulkin-W.]
		A category $\C$ with coproducts and an algebraic WFS has:
		\begin{itemize}
			\item a cylinder functor $\Cyl \colon \C \to \C$
			\item faces $\partial_0, \partial_1 \colon \Id \To \Cyl$ 
			\item degeneracies $\sigma \colon \Cyl \To \Id$ 
			\item connections $\gamma_0, \gamma_1 \colon \Cyl^2 \To \Cyl$
		\end{itemize}
		satisfying the identities for faces, degeneracies and connections.
	\end{proposition}
	\pause \vfill
	Each $X \in \C$ gives a co-cubical object with connections:
	\[
		\begin{tikzcd}
			X \ar[r, shift left = 2] \ar[r, shift right = 2] & \Cyl X \ar[l] \ar[r, shift left = 2] \ar[r, shift left = 6] \ar[r, shift right = 2] \ar[r, shift right = 6] & \Cyl^2 X \ar[l, shift left = 4] \ar[l] \ar[l, shift right = 4] \ar[r, phantom, "\cdots"] & \cdots
		\end{tikzcd}
	\]
\end{frame}

\begin{frame}{Cubical enrichment from algebraic WFS}
	\begin{corollary}
		A category $\C$ with coproducts and an algebraic WFS is enriched over cubical sets, with
		\[
			\C(X,Y)_n := \C(\Cyl^n X, Y).	
		\]
	\end{corollary}
	\pause \vfill
	\begin{theorem}[Garner '09]
		Let $\C$ be a cocomplete category satisfying a `smallness' condition (e.g. locally presentable).
		Then any cofibrantly generated WFS can be upgraded to an algebraic WFS.
	\end{theorem}
	\pause \vfill
	\begin{corollary}
		Any category satisfying the above conditions is enriched over cubical sets.
	\end{corollary}
\end{frame}

\begin{frame}{Final remarks}
	\begin{itemize}[<+->]\setlength{\itemsep}{15pt}
		\item The dual version also holds (with products \& path objects)
		\item If $\C$ is an \emph{algebraic model category} (Riehl '11):
			\begin{itemize}[<+->]
		 		\item We get two enrichments: how are they related?
				\item Is $\C$ an \emph{enriched model category}?
			\end{itemize}
		\item By quotienting cubes, can we make $\C$ \emph{simplicially} enriched?
	\end{itemize}	
\end{frame}

\begin{frame}
  \vfill
  \centering
  \begin{beamercolorbox}[sep=8pt,center,shadow=true,rounded=true]{title}
    \usebeamerfont{title}Thank you!\par%
  \end{beamercolorbox}
  % \begin{center}
  % Questions/comments?
  % \end{center}
  \vfill
\end{frame}

\begin{frame}
  \vfill
  \centering
  \begin{beamercolorbox}[sep=8pt,center,shadow=true,rounded=true]{title}
    \usebeamerfont{title}Background on functorial and algebraic WFS\par%
  \end{beamercolorbox}
  % \begin{center}
  % Questions/comments?
  % \end{center}
  \vfill
\end{frame}

\begin{frame}[fragile]{Functorial factorizations}
	A \emph{functorial factorization} on a category $\C$ is a functor
	\[
		\C^\ii \xto{\quad \quad} \C^\iii
	\]
	\[
		\begin{tikzcd}
			x \ar[d, "f"']
			\\
			y
		\end{tikzcd}
		\xmapsto{\quad \quad}
		\begin{tikzcd}
			x \ar[d, "Lf"]
			\\
			Ef \ar[d, "Rf"]
			\\
			y 
		\end{tikzcd}
	\]	
	that is a section of the composition functor.
	\pause \vfill
	This gives rise to functors $L, R \colon \C^\ii \to \C^\ii$ and $E \colon \C^\ii \to \C$.
	\pause \vfill
	A WFS $(\L, \R)$ is \emph{functorial} if it has a functorial factorization with
	\[
		Lf \in \L \quad \quad \text{and} \quad \quad Rf \in \R.
	\]
\end{frame}

\begin{frame}[fragile]{Functorial factorizations}
	The endofunctor $R \colon \C^\ii \to \C^\ii$ is \emph{pointed}: \\~\\
	\pause 
	i.e. there is a natural transformation $\eta \colon \Id \To R$
	\[
		\eta_f =
		\begin{tikzcd}
			\cdot \ar[d, "f"'] \ar[r, "Lf"] & \cdot \ar[d, "Rf"]
			\\
			\cdot \ar[r, equals] & \cdot
		\end{tikzcd}
	\]
	\pause
	We can go further and ask for $\mu \colon RR \To R$ making $R$ a \emph{monad}.
	\pause \vfill
	Dually: \\~\\

	$L \colon \C^\ii \to \C^\ii$ is \emph{co-pointed}, and we may ask if $L$ is a \emph{comonad}.
\end{frame}

\begin{frame}[fragile]{Algebraic weak factorization systems}
	\begin{definition}[Grandis-Tholen `06, Garner `07]
		An \emph{algebraic} WFS is a functorial WFS along with
		\begin{itemize}
			\item $\delta \colon L \To LL$ making $L$ a comonad
			\item $\mu \colon RR \To R$ making $R$ a monad
			\item such that $(\delta, \mu) \colon LR \To RL$ is a distributive law
		\end{itemize}
	\end{definition}
		\[	
			\begin{tikzcd}
				\cdot \ar[r, equals] \ar[d, "Lf"'] & \cdot \ar[d, "LLf"]
				\\
				\cdot \ar[r, "\delta_f", red] \ar[d, "LRf"'] & \cdot \ar[d, "RLf"]
				\\
				\cdot \ar[r, "\mu_f"', red] \ar[d, "RRf"'] & \cdot \ar[d, "Rf"]
				\\
				\cdot \ar[r, equals] & \cdot
			\end{tikzcd}
		\]	
\end{frame}

\begin{frame}[fragile]{$R$-algebras and $L$-coalgebras}
	Since $R$ is a monad, we can talk about $R$-algebras:
	maps $g$ equipped with an action $\alpha \colon Rg \to g$.
	\pause \vfill
	Dually, \emph{$L$-coalgebras} have a co-action $\kappa \colon f \to Lf$.
	\pause
	\[
		\begin{tikzcd}
			\cdot \ar[r,equals] \ar[d, "f"'] & \cdot \ar[r, equals] \ar[d, "Lf"] & \cdot \ar[d, "f"]
			\\
			\cdot \ar[r, "\kappa", red] & \cdot \ar[r, "Rf"] & \cdot
		\end{tikzcd}
		\quad \quad \quad		
		\begin{tikzcd}
			\cdot \ar[d, "g"']  \ar[r, "Lg"] & \cdot \ar[r, "\alpha", red] \ar[d, "Rg"] & \cdot \ar[d, "g"]
			\\
			\cdot \ar[r, equals] & \cdot \ar[r, equals] & \cdot		
		\end{tikzcd}
	\]
	\pause
	This gives \emph{canonical} lifts of $L$-coalgebras against $R$-algebras!
	\[
		\begin{tikzcd}
			\cdot \ar[r] \ar[d, "L\text{-coalg} \; \ni f"'] & \cdot \ar[d, "g \in \; R\text{-alg}"]
			\\
			\cdot \ar[r] \ar[ur, dashed, red] & \cdot
		\end{tikzcd}
		\quad \quad = \quad \quad
		\begin{tikzcd}
			\cdot \ar[r] \ar[d, "Lf"'] & \cdot \ar[d, "Lg"]
			\\
			\cdot \ar[r, red] \ar[d, "Rf"'] & \cdot \ar[d, "Rg"] \ar[u, bend left, "\alpha", red]
			\\
			\cdot \ar[r] \ar[u, bend right, "\kappa"', red] & \cdot
		\end{tikzcd}
	\]
\end{frame}

\begin{frame}[fragile]{Free algebras}
	We can do even better for \emph{free} $R$-algebras: $Rf$ for any $f$, with action given by $\mu_f \colon RRf \to Rf$.
	\pause
	\vfill
	For any commuting square where $g$ is an $R$-algebra: 
	\[
		\begin{tikzcd}
			\cdot \ar[r] \ar[d, "f"'] & \cdot \ar[d, "g \in \; R\text{-alg}"]
			\\
			\cdot \ar[r] & \cdot
		\end{tikzcd}
	\]
	there is a \emph{unique} $R$-algebra map $\varphi \colon Rf \to g$:
	\[
		\begin{tikzcd}
			\cdot \ar[rr] \ar[dd, "f"'] \ar[dr, "Lf"] & & \cdot \ar[dd, "g"]
			\\
			& \cdot \ar[dl, "Rf"] \ar[ur, dashed, "\varphi", red] &
			\\
			\cdot \ar[rr] & & \cdot
		\end{tikzcd}
	\]
	\pause
	Warning: this is only unique \emph{as an $R$-algebra map}.
\end{frame}


\end{document}	